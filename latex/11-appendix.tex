\newpage

\thispagestyle{empty}
\section*{Prilozi} % (fold)
\addcontentsline{toc}{section}{Prilozi}
\label{sec:Prilozi}

\subsubsection*{Izlistanje koda} % (fold)
\label{ssub:Izlistanje koda}

\begin{lstlisting}[caption={}]
/* Osnovna ideja:
 * usporedivanje slike predloska (znak) sa svakom slicicom i pronalazak.
 * Tamo gdje se predlozak i slicica podudaraju proglasimo znakom i 
 * nacratom pravokutnik.
 * Takav pristup daje puno lazno pozitivnih rezultata.
 * Taj problem pokusavamo rjesiti racunanjem udaljenosti izmedu 
 * trenutne i prijasnjih slicica, te postavljamo prag ispod kojeg
 * pronadene lokacije proglasavamo znakom.
 */ 

// OpenCV zaglavlja za koristenje openCV funkcija
#include "opencv2/highgui/highgui.hpp"
#include "opencv2/imgproc/imgproc.hpp"
#include <iostream>
#include <stdio.h>
#include <math.h>

using namespace std;
using namespace cv;

// deklaracija funkcije za izracun udaljenosti izmedu dvije tocke
int calcPointDist (Point maxLoc, Point prevMaxLoc);

// vektor tocaka
vector<Point> vPoint;
Point minLoc; Point maxLoc; Point prevMaxLoc;

int main( int argc, char** argv )
{
    // kreiranje objekta cap za ucitavanje videa
    VideoCapture cap("video/znakich2.mp4");
    if(!cap.isOpened())  // check if we succeeded
        return -1;

    /* kreiranje objekata, 
     * #rect# za definiranje regije (sredina-desno) u kojoj trazimo 
     * znak kako ne bi pretrazivali cijelu sliku
     * #frame# za trenutnu slicicu iz videa
     * #roi# za slicicu regije definirane iz "rect" objekta
     * #groi# za sivu slicicu "roi-a"
     * #results# za spremanje rezultata pronalaska znaka 
    */
    Rect rect;
    Mat frame, roi, groi, results;
    Mat znak1, znak2, znak3, znak4;
    
    // ucitavanje izrezanih znakova u razlicitim velicinama
    // trenutno se koristi samo znak2 
    znak1 = imread ("roi/01_roi.png");      
    znak2 = imread ("roi/02_roi.png");
    znak3 = imread ("roi/03_roi.png");
    znak4 = imread ("roi/04_roi.png");
    // prebacivanje znak2 u sliku sivih nijansi
    cvtColor (znak2, znak2, CV_BGR2GRAY);
    
    // ucitavanje slicice iz videa u frame
    // postavljanje regije interesa u roi
    cap >> frame;
    rect = Rect (600, 150, 480, 120);
    roi = frame(rect);
    
    // postavljanje velicine results ovisno o velicini roi i znak2
    int resultRows, resultCols;
    resultRows = roi.rows - znak2.rows + 1;
    resultCols = roi.cols - znak2.cols + 1;
    results.create (resultRows, resultCols, CV_32FC1);

    // pretvaranje roi u greyscale
    cvtColor(roi, groi, CV_BGR2GRAY);      
    // trazenje znaka u greyscale groi i spremanje u results
    matchTemplate (groi, znak2, results, 5);
    // normaliziranje rezultata
    normalize (results, results, 0, 1, NORM_MINMAX, -1);
    // traznje lokacije najveceg rezultata
    double minVal; double maxVal;
    minMaxLoc (results, &minVal, &maxVal, &minLoc, &maxLoc, Mat() );
    // spremanje rezultata u vektor tocaka
    vPoint.push_back (maxLoc);
    
    // Postavljanje prozora za prikaz frame-a
    // namedWindow("frame",1);
    
    // inicijalizacija varijable za udaljenost izmedu tocaka rezultata
    int dist = 0;
    
    // Iznad je za prvi frame da se moze usporediti
    // Sada ponavlja beskonacno za sve ostale slicice
    for(;;) {        
        // ucitaj frame, postavi roi, prebaci u greyscale
        cap >> frame;              
        //imshow("frame", frame);
        roi = frame(rect);
        cvtColor(roi, groi, CV_BGR2GRAY);
        
        // pronadi podudarani template
        // normaliziraj rezultate
        // pronadi lokacije tocaka s najvecim vrijednostima
        // spremi lokacije tocaka u vektor
        matchTemplate (groi, znak2, results, 5);
        normalize (results, results, 0, 1, NORM_MINMAX, -1);
        //threshold (results, results, 0.8, 1, THRESH_BINARY);
        minMaxLoc (results, &minVal, &maxVal, &minLoc, &maxLoc, Mat() );
        vPoint.push_back (maxLoc);
        int size = vPoint.size();
        
        // Za svaki rezultat veci od 0.85 
        if (maxVal > 0.75) {
            // racunaj udaljenost izmedu dva rezultata
            // Ako je vektor tocaka manji od 10 clanova
            if (size < 10) {
                // racunaj udaljenost izmedu trenutne i prve tocke
                dist = calcPointDist (maxLoc, vPoint.at(0));
                // ako je udaljenost manja od 10 iscrtaj znak
                if (dist < 10) {
                    rectangle (groi, maxLoc, Point(maxLoc.x + znak2.cols, 
                    maxLoc.y + znak2.rows), Scalar::all(0), 2, 18, 0 );
                    //vPoint.push_back (maxLoc);    
                }
            }
            // Ako je vektor tocaka veci od 10 clanova
            else {
                // racunaj udaljenost izmedu trenutnog i sedmog prije 
                dist = calcPointDist (maxLoc, vPoint.at(size-7));  
                // iscrtaj pravokutnik ako je udaljenost manja od 15
                if (dist < 15) {
                    rectangle (groi, maxLoc, Point(maxLoc.x + znak2.cols, 
                    maxLoc.y + znak2.rows), Scalar::all(0), 2, 18, 0 );
                    // ponovi istu tocku u vektor kako bih imali duze pracanje znaka
                    vPoint.push_back (maxLoc);  
                }
                else {
                // ponovi istu tocku u vektor kako bih imali duze pracanje znaka
                vPoint.push_back (maxLoc);
                }
            }
        }
        cout << dist << endl;
        
        // prikazi greyscale sliku regije i rezultate
        imshow("groi", groi);
        imshow("results", results);
        
        char c = cv::waitKey(15);
        /// Kontrola programa         
        // *p* zaustavi i pokreni
        // *c* pokreni ispocetka
        switch( c ) {
            case 'c': 
                cap.set(CV_CAP_PROP_POS_FRAMES,0); 
                break;    
            case 'p': 
                pause:
                char d; d = cv::waitKey(0);
                if (d =='p') break;
                else goto pause;
            case 27:
                return 0;
        }        
    }
    return 0;
}

int calcPointDist (Point maxLoc, Point prevMaxLoc)
{
    int dist = 0;
    dist = sqrt ((maxLoc.x - prevMaxLoc.x) * (maxLoc.x - prevMaxLoc.x) +
                 (maxLoc.y - prevMaxLoc.y) * (maxLoc.y - prevMaxLoc.y));    
    cout << dist << endl;
    return dist;
}
\end{lstlisting}

% subsubsection Izlistanje koda (end)

% section Prilozi (end)
