\newpage
\setcounter{figure}{0}

\section{Zaključak} % (fold)
\label{sec:Zaključak}

Cilj ovog završnog rada razrada je i implementacija jedne od
metodologija računalnog vida za pronalazak prometnog znaka u pokretnoj
slici. Izabrana metoda, algoritam usporedba s predloškom, implementirana
je u programu video-template-matching. Metoda je jednostavna ali u
idealnim uvijetima daje relativno dobre rezultate, naravno takva
implementacija se ne može koristiti u stvarnim uvijetima. Postoji puno
mjesta za unaprijeđenje metode. Trenutno je program ograničen na
prepoznavanje samo jedne vrste znaka na točno određenoj regiji interesa.
Nije rješenje proširivanje regije interesa i uključivanje pretraživanja
dodatnih tipova znakova jer se tada multiplicira težina obrade podataka
te program ne bi radio u stvarnom vremenu. Praktično rješenje bi trebalo
tražiti u drugim metodologijama računalnog vida npr. strojno učenje.
Jedan od prijedloga poboljšanja metode je evaluacija odnosno testiranje
pronalaska znaka. Naime programu bi se mogao dodati brojač pronađenih
znakova te se takav brojač može usporediti s brojem sličica na kojima se
znak pojavljuje.

% section Zaključak (end)
