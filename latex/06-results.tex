\newpage
\setcounter{figure}{0}

\section{Rezultati} % (fold)
\label{sec:Rezultati}

U ovom poglavlju prikazani su rezultati rada razvijenog programa za
prepozavanje statičnog znaka iz snimke prometnice vozila u pokretu.  Za
ispitivanje funkcionalnosti metode odabrana je jedna scena voženje i
jedan znak kojeg metoda treba pronaći. Znak je prikazan na
slici~\ref{fig:znak}, a scena vožnje je snimljena na osječkoj
obilaznici i prikazana na slici~\ref{fig:scena.png}.

\begin{figure}[h]
\centering
\includegraphics[scale=0.5]{figures/znak.png}
\caption{Prikaz korištenog znaka - obvezan smjer kretanja u desno}
\label{fig:znak}
\end{figure}

\begin{figure}[h]
\centering
\includegraphics[scale=0.3]{figures/scena.png}
\caption{Prikaz testirane scene voznje}
\label{fig:scena.png}
\end{figure}

\newpage
\subsection{Prikaz rezultata} % (fold)
\label{sub:Prikaz rezultata}

Rezultati su prikazani na slikama s dva prozora. U prvom prozoru 
prikazana je regija interesa sa scene unutar koje algoritmom
usporedbe tražimo znak. U drugom prozornu prikazana je matrica rezultata
koju metoda koristi za pronalazak znaka. Postoji nekoliko tipova
rezultata: pozitivni, negativni, lažno pozitivni.
Svi tipovi rezultata su predstavljeni dalje u tekstu.
Slika~\ref{fig:01} prikazuje sličicu iz videa sa scene na kojoj nema
znaka niti ga je metoda/program našao što je pozitivan rezultat.

\begin{figure}[h]
\centering
\includegraphics[scale=0.5]{figures/01.png}
\caption{Prikaz sličice videa i rezultata - pozitivan rezultat}
\label{fig:01}
\end{figure}

Lažno pozitivni rezultat prikazuje slika~\ref{fig:02} na kojoj je metoda
pronašala znak gdje nije trebala. Takvi rezultati su očekivani ali nisu
poželjni te ih se pokušalo smanjiti na što manji broj metodom opisanom 
u podpoglavlju~\ref{ssub:Eliminacija lažno pozitivnih rezultata}

\begin{figure}[h]
\centering
\includegraphics[scale=0.5]{figures/02.png}
\caption{Prikaz sličice videa i matrice rezultata - lažno pozitivan
rezultat}
\label{fig:02}
\end{figure}

\newpage
\begin{figure}[h]
\centering
\includegraphics[scale=0.5]{figures/03.png}
\caption{Prikaz sličice videa i matrice rezultata - pozitivan rezultat }
\label{fig:03.png}
\end{figure}

Na slikama~\ref{fig:03.png}, \ref{fig:04.png} i \ref{fig:05.png} vidi se
da je metoda uspješno pronašla znak na različitim udaljenostima odnosno
veličinama znaka iako se koristila samo jedna veličina znaka za
pronalazak. Metoda odnosno program bih se mogao unaprijediti tako da se
ugradi uspoređivanje s različitiim veličinama predloška odnosno znaka.

\begin{figure}[h]
\centering
\includegraphics[scale=0.5]{figures/04.png}
\caption{Prikaz sličice videa i matrice rezultata - pozitivan rezultat}
\label{fig:04.png}
\end{figure}

Matematička metoda korištena u algoritmu prikazuje pozitivne rezultate
bijelom bojom te se na slikama~\ref{fig:04.png} i \ref{fig:05.png} jasno
mogu vidjeti "žarišta" u matrici rezultata.

\begin{figure}[h]
\centering
\includegraphics[scale=0.5]{figures/05.png}
\caption{Prikaz sličice videa i matrice rezultata - pozitivan rezultat}
\label{fig:05.png}
\end{figure}

\begin{figure}[h]
\centering
\includegraphics[scale=0.5]{figures/06.png}
\caption{Prikaz sličice videa i matrice rezultata - pozitivan rezultat}
\label{fig:06.png}
\end{figure}

\begin{figure}[h]
\centering
\includegraphics[scale=0.5]{figures/07.png}
\caption{Prikaz sličice videa i matrice rezultata - pozitivan rezultat}
\label{fig:07.png}
\end{figure}


% subsection Prikaz rezultata (end)
% section Rezultati (end)
