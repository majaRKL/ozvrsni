\newpage
\thispagestyle{empty}

\section*{Sažetak} % (fold)
\addcontentsline{toc}{section}{Sažetak}
\label{sec:Sažetak}

U ovom radu implementirana je i prikazana jedna od metodologija
računalnog vida za pronalazak prometnog znaka u pokretnoj slici.
Napisani program upotrebljava algoritam usporedbe s predloškom za
pronalazak znaka u zadanoj pokretnoj slici. Program se oslanja na mnoge
funkcije iz OpenCV biblioteke za pronalazak znaka. Rezultati rada
programa su predstavljeni, opisane su prednosti i nedostatci odabrane
metode te su dani prijedlozi za daljnja unaprijeđenja.
\\[0.5cm]

\noindent\textbf{Ključne riječi:} OpenCV, procesiranje videa, usporedba
s predloškom, obrada slike.

\section*{Abstract} % (fold)
\label{sec:Abstract}

The aim of this study was to implement one of computer vision
methodologies for finding traffic sign in motion picture. Developed
program uses template matching algorithm for finding sign in given
motion picture. Program is based on many OpenCV library functions. 
Results of developed program are presented, advantages and
disadvantages of used method are described and also propsitions for
further improvements are given.
\\[0.5cm]

\noindent\textbf{Keywords:} OpenCV, Video processing, template matching,
image processing.

% section Sažetak (end)
