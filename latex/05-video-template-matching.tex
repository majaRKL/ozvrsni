\newpage
\setcounter{figure}{0}

\section{Prepoznavanje prometnog znakovlja u pokretnoj slici} % (fold)
\label{sec:Prepoznavanje prometnog}

\subsection{Prepoznavanje prometnog znakovlja upotrebom metode usporedbe
s predloškom} % Prepoznavanje (fold)
\label{sub:Prepoznavanje prometnog znakovlja upotrebom metode uspore}

Prepoznavanje prometnog znaka u pokretnoj slici implementirano je u
programu nazvanom \texttt{video-template-matching}. Program se u
potpunosti oslanja na biblioteku OpenCV koja je opisana u
podpoglavlju~\ref{sub:Biblioteka OpenCV} Program se sastoji od pet
logičnih cijelina. 

\begin{itemize}
    \item Učitavanje videa vožnje gradom i učitavanje znaka.
    \item Postavljanje regije interesa na učitanom videu.
    \item Prebacivanje svake sličice iz regije interesa u sličicu sivih
        tonova. 
    \item Obrađivanje takve regije interesa metodom usporedbe s učitanim
        znakom koji je isto slika sivih tonova.
    \item Obrađivanje rezultata, proglašavanje i iscrtavanje pronađenog
        znaka nad svakom sličicom.
\end{itemize}

\begin{figure}[h]
\centering
\includegraphics[scale=0.4]{figures/dijagramtoka.pdf}
\caption{Dijagram toka programa video-template-matching}
\label{fig:dijagramtoka.pdf}
\end{figure}


\newpage
\subsubsection{Učitavanje videa vožnje i učitavanje znaka} % (fold)
\label{ssub:Učitavanje videa vožnje i učitavanje znaka}


\begin{lstlisting}[label=lstUcit,caption={Izvorni kod za učitavanje
videa i znaka}]
#include "opencv2/imgproc/imgproc.hpp"

int main (int argc, char *argv[])
{
    // kreiranje objekta cap za ucitavanje videa
    VideoCapture cap("video/znakich2.mp4");
    if(!cap.isOpened())  // provjera uspjeha ucitavanja
        return -1;

    // kreiranje objekta Mat za spremanje znakova
    Mat znak1, znak2, znak3, znak4;

    // ucitavanje izrezanih znakova u razlicitim velicinama
    // trenutno se koristim samo znak2 
    znak1 = imread ("roi/01_roi.png");		
    znak2 = imread ("roi/02_roi.png");
    znak3 = imread ("roi/03_roi.png");
    znak4 = imread ("roi/04_roi.png");

}
\end{lstlisting}

Izlistanje koda~\ref{lstUcit} prikazuje primjer učitavanja videa i
učitavanje znaka upotrebom klase \texttt{VideoCapture} i funkcije
\texttt{imread} koji su uključeni dodavanjem biblioteke
\texttt{imgproc.hpp}. Kreiranom objektu \texttt{cap} predana je putanja
do videa kojeg treba učitati. Ukoliko video nije uspješno učitan program
završava i vraća \texttt{-1}. Kreiranim objektima \texttt{znak1},
\texttt{znak2}, \texttt{znak3}, \texttt{znak4} pridjeljene su slike
učitane upotrebom funkcije \texttt{imread} kojoj je predana putanja do
slike koju treba učitati.

% subsubsection Učitavanje videa vožnje i učitavanje znaka (end)

\subsubsection{Postavljanje regije interesa} % (fold)
\label{ssub:Postavljanje regije interesa}


\begin{lstlisting}[label=lstRoi,caption={Izvorni kod za postvljanje
regije interesa}]
	// prebacivanje znak2 u sliku sivih nijansi
	cvtColor (znak2, znak2, CV_BGR2GRAY);
	
	// ucitavanje slicice iz videa u frame
	// postavljanje regije interesa u roi
	cap >> frame;
	rect = Rect (600, 150, 480, 120);
    roi = frame(rect);
	
\end{lstlisting}

Izlistanje koda~\ref{lstRoi} prikazuje upotrebu funkcije
\texttt{cvtColor} kojom se učitani znak prebacuje u sliku sivih tonova.
Zatim se učitava sličica u objekt \texttt{frame} iz objekta
\texttt{cap}. Tada se kreira objekt \texttt{rect} pomoću funkcije
\texttt{Rect()} kojom definiramo pravokutnik koordinatama gornjeg
lijevog kuta te širinom i visinom. Takav pravokutnik upotrebljava se za
definiranje regije interesa nad učitanom sličicom odnosno objektom
\texttt{frame} te se regija interesa sprema u objekt \texttt{roi}.
Regija interesa se kasnije upotrebljava u algoritmu usporedbe s
predloškom.

% subsubsection Postavljanje regije interesa (end)

\newpage
\subsubsection{Pozivanje algoritma usporedbe s predloškom} % (fold)
\label{ssub:Pozivanje algoritma usporedbe s predloškom}

\begin{lstlisting}[label=lstTemp,caption={Izvorni kod pozivanja
algoritma usporedbe s predloškom}]
	// postavljanje velicine results ovisno o velicini roi i znak2
	int resultRows, resultCols;
	resultRows = roi.rows - znak2.rows + 1;
	resultCols = roi.cols - znak2.cols + 1;
	results.create (resultRows, resultCols, CV_32FC1);

    // pretvaranje roi u greyscale
    cvtColor(roi, groi, CV_BGR2GRAY);      
    // trazenje znaka u greyscale groi i spremanje u results
    matchTemplate (groi, znak2, results, 5);
    // normaliziranje rezultata
    normalize (results, results, 0, 1, NORM_MINMAX, -1);
    // traznje lokacije najveceg rezultata
    double minVal; double maxVal;
    minMaxLoc (results, &minVal, &maxVal, &minLoc, &maxLoc, Mat() );
    // spremanje rezultata u vektor tocaka
    vPoint.push_back (maxLoc);

\end{lstlisting}

Prije pozivanja algoritma usporedbe s predloškom implementiranog u
funkciji \texttt{matchTemplate()} potrebno je kreirati matricu/sliku u
koju će funkcija spremati rezultate. Matrica \texttt{results} se kreira
ovisno o veličini matrica \texttt{roi} i \texttt{znak2} kao što
prikazuje izlistanje koda~\ref{lstTemp}. Kako su svi parametri koje
funkcija \texttt{matchTemplate()} kreirani, slijedi njeno pozivanje.
Rezultat rada algoritma je spremljen u \texttt{results} te se takvi
rezultati normaliziraju korištenjem funkcije \texttt{normalize()}. Zatim
slijedi lociranje najvećih i najmanjih vrijednosti piksela u rezultantnoj
matrici upotrebom funkcije \texttt{minMaxLoc()}. Korištena metoda
usporedbe sprema najbolje rezultate kao maksimalne vrijednosti i zato se
\texttt{maxLoc} sprema u vektor točaka za daljnju obradu.


% subsubsection Pozivanje algoritma usporedbe s predloškom (end)

% subsection Prepoznavanje (end)

% section Prepoznavanje prometnog (end)
