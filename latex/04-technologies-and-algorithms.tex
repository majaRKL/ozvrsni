\newpage
\setcounter{figure}{0}

\section{Pregled korištenih tehnologija i algoritama} % (fold)
\label{sec:Tehnologija i teorija}

\subsection{Biblioteka OpenCV} % (fold)
\label{sub:Biblioteka OpenCV}

OpenCV~\cite{opencv_library} (engl. \textit{Open Source Computer Vison})
je biblioteka funkcija odnosno algoritama za matematičku obradu slike i
računalni vid.  Objavljen je pod BSD licencom i stoga je slobodan za
akademske i komercijalne uporabe. Sadrži C++, C, Python i Java sučelja i
podržava Windows, Linux, Mac OS, iOS i Android platforme. OpenCV je
dizajniran za računalnu učinkovitost, sa snažnim naglaskom na stvarno
vremenske aplikacije. Biblioteka je napisana u optimiziranom C/C++ kodu
te može iskoristiti prednosti višejezgrenog procesiranja. Ako se koristi
s OpenCL (engl. \textit{Open Computing Language}) bibliotekom OpenCV
može iskoristi prednosti hardverskog ubrzanja osnovne računalne
platforme.

Usvojen diljem svijeta, OpenCV ima zajednicu korisnika veću od 47 tisuća
ljudi i procijenjeni broj preuzimanja prelazi sedam milijuna. Primjene
variraju od interaktivne umjetnosti, do inspekcija mina, spajanja
internetskih mapa i napredne robotike.

\begin{figure}[h]
\centering
\includegraphics[scale=0.8]{figures/opencv.jpg}
\caption{OpenCV logo}
\label{fig:opencv.svg}
\end{figure}

\newpage
\subsection{Radna okolina} % (fold)
\label{sub:Radna okolina}

U ovom podpoglavlju je dan kratak pregled korištene radne okoline. Pod
radnom okolinom se misli na operacijski sustav, upotrebljavane alate i
programske biblioteke. Tako su izabrane slijedeće komponente:

\begin{description}
  \item[OS:] Ubuntu 12.04
  \item[Biblioteka:] OpenCV 2.4.6
  \item[Prevodioc:] GCC 4.6.3 
  \item[Uređivač teksta:] Geany
\end{description}

Ubuntu 12.04 je izabran zbog stabilnosti, jednostavnog podešavanja i 
dostupnosti velike količine već priprmljenih programskih paketa.
OpenCV 2.4.6 je zadnja stabilna verzija u trenutnku pisanja. Odabrana
je jer se lagano prevodi i instalira na odabranoj verziji Ubuntu OSa.
GCC 4.6.3 je zadna verzija koja dolazi s Ubuntu 12.04 distribucijom.
Geany je uporijebljen zbog svoje jednostavnosti i integracije s GCC
prevoditeljem.
\\


\begin{figure}[!htb]
\minipage{0.32\textwidth}
    \includegraphics[width=\linewidth]{figures/ubuntu.png}
\endminipage\hfill
\minipage{0.32\textwidth}
    \includegraphics[width=\linewidth]{figures/gcc.jpeg}
\endminipage\hfill
\minipage{0.32\textwidth}%
    \includegraphics[width=\linewidth]{figures/geany.jpg}
\endminipage
\caption{Prikaz logotipa upotrebljenih tehnologija}
\end{figure}

% subsection Radna okolina (end)

\newpage
% subsection Biblioteka OpenCV (end)

\subsection{Algoritam usporedbe predloškom} % (fold)
\label{sub:Algoritam usporedbe predloškom}

Postoji više metoda odnosno algoritama koje se koriste za prepoznavanje
različitih objekata na slici. Jedna od njih je algoritam usporedbe s
predloškom (engl. \textit{Template Matching}). U ovom podpoglavlju dan
je primjer korištenja takve metode na mirnoj slici. 

Algoritam usporedbe predloškom je metoda za pronalaženje područja na
slici koji odgovaraju odnosno su slični uspoređenom predlošku. 

Princip rada se zansniva na dvije komponente:
\begin{itemize}
    \item \textbf{Izvorna slika (I)} - slika u kojoj se pronalazi
        odgovarajuća slika predloška.
    \item \textbf{Slika predloška (T)} - manja slika odnosno uzorak koji
        se traži na izvornoj slici.
\end{itemize}

Cilj metode je otkriti najviše odgovarajuća mjesta. Na
slici~\ref{fig:tm1.jpg} vidi se upotreba izvorne slike, slike predloška
i prikaz krajnjeg rezultata metode usporedbe predloškom.

\begin{figure}[h]
\centering
\includegraphics[scale=1]{figures/tm1.jpg}
\caption{Prikaz izvorne slike, slike predloška i krajnjeg rezultata
metode usporedbe predloškom}
\label{fig:tm1.jpg}
\end{figure}

Identificiranje odgovarajućeg područja, radi se uspoređivanjem slike
predloška s izvornom slikom. Pomicanjem predloška piksel po piksel po
izvornoj slici s lijeva na desno, od gore prema dolje kao što je
prikazano na slici~\ref{fig:tm2.jpg}. Na svakoj lokaciji, računa se
vrijednosti koja predstavlja koliko dobro predložak odgovara izvornoj
slici, odnosno koliko su slični.

\begin{figure}[h]
\centering
\includegraphics[scale=0.5]{figures/tm2.jpg}
\caption{Prikaz pomicanja predloška po izvornoj slici}
\label{fig:tm2.jpg}
\end{figure}

For each location of T over I, you store the metric in the result matrix
(R). Each location (x,y) in R contains the match metric:

Za svaku lokaciju piksela u predlošku T preko slike I, pohranjuju se
vrijednosti u matricu rezultata R. Svaka lokacija (x, y) u matrici R
sadrži vrijednosti koja govori koliko je na toj lokaciji predložak
sličan izvornoj slici. 

Slika~\ref{fig:tm3.jpg} prikazuje rezultate R pomicanja predloška po
izvornoj slici. Najsvjetlije mjesta pokazuju odgovarajuće
vrijednosti. Kao što se vidi, mjesto obilježeno crvenim krugom 
je ono s najvišom vrijednosti.  


\begin{figure}[h]
\centering
\includegraphics[scale=0.8]{figures/tm3.jpg}
\caption{}
\label{fig:tm3.jpg}
\end{figure}
% subsection Algoritam usporedbe predloškom (end)

% section Tehnologija i teorija (end)
