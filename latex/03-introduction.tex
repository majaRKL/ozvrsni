\newpage

\setcounter{page}{1}
\setcounter{figure}{0}
\section{Uvod}% (fold)
\label{sec:Uvod}

Prometni znakovi su sastavni dio prometne okoline. Osmišljeni su da
reguliraju tijek vozila, daju specifične informacije o stanju u prometu
ili upozoravaju na nepredviđena stanja na cesti. Percepcija i brzo
prepoznavanje prometnih znakova je ključna za vozačevu sigurnost. Javne
institucije odgovorne za prometnu infrastrukturu i održavanje
postavljaju znakove na stupove pored ceste, iznad ceste i na ostala
mjesta gdje je osigurana njihova vidljivost. Međutim u određenim
uvijetima poput jakog sunca ili slabe vidljivosti zbog kiše percepcija
može biti otežana. Isto tako razni psihološki razlozi mogu utjecati na
vozačevu percepciju. Iz navedenih razloga automatska detekcija i
prepoznavanje je postala prirodan smjer istraživanja nakon što su
računala postala dovoljno brza za obrađivanje videa u stvarnom vremenu.   

Prepoznavanje prometnog znakovlja u pokretnoj slici ima niz primjena.
Autoindustrija je najočitiji primjer, ukoliko se radi o online detekciji
i prepoznavanju znakova informacije o pronađenim znakovima mogu se
upotrijebiti kao dodatne sigurnosne obavijesti vozaču. Isto tako
detekcija znakova je iznimno bitna za samovozeće automobile gdje je to
jedna u nizu metoda koja se koristi za autonomnu vožnju. Osim toga
offline detekcija znakova se može koristi za inspekciju sigurnosti na
cestama npr. provjeravanje ispravne vidljivosti, istrošenosti,
nepravilnog položaja znakova itd. 

Ovaj rad se sastoji iz tri dijela. Prvi dio odnosi se na pregled
korištenih tehnologija i algoritama. Prvo je predstavljena OpenCV
biblioteka, zatim je opisana korištena radna okolina te je dan pregled
upotrebljenog algoritma za pronalazak znaka. Drugi dio je praktični
dio i prikazuje implementaciju spomenutih tehnologija. Treći dio rada
prikazuje rezultate i ispituje kvalitetu i funkcionalnost odabrane
metode.


\newpage
\subsection{Zadatak diplomskog rada} % (fold)
\label{sub:Zadatak diplomskog rada}
Veliki napori ulažu se u prepoznavanje objekata snimljene slike uporabom
računalnog vida. Isto tako, u autoindustriji, uloženi su veliki napori
primjene računalnog vida u prepoznavanju objekata na prometnici ispred
vozila u pokretu. Zadatak ovog rada jest razraditi i implementirati
metodu prepoznavanja objekata statičnog prometnog znakovlja iz snimke
prometnice vozila u pokretu uporabom metodologija koje se oslanjaju na
metode računalnog vida.

% subsection Zadatak diplomskog rada (end)
% section Uvod (end)
